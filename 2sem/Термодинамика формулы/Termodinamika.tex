\documentclass[12pt]{article}
\usepackage[a4paper]{geometry}
% See geometry.pdf to learn the layout options. There are lots.
\geometry{a4paper}
\usepackage{listings}
\usepackage[cm]{fullpage}
\usepackage{layout}
\usepackage{amssymb,amsmath,amsfonts,latexsym,dsfont}
\usepackage{ upgreek }
\usepackage{xcolor}
\usepackage{titlesec}
\usepackage[warn]{mathtext}
\usepackage[T1,T2A]{fontenc}
\usepackage[utf8]{inputenc}
\usepackage{fancyhdr}
\usepackage[english,bulgarian,ukrainian,russian]{babel}
%\titleformat{\section}[block]{\color{black}\Large\bfseries\filcenter}{}{1em}{}
\setcounter{secnumdepth}{0}
\renewcommand{\le}{\leqslant} 
\renewcommand{\ge}{\geqslant }

% ... or a4paper or a5paper or ... 
%\geometry{landscape} % Activate for rotated page geometry
%\usepackage[parfill]{parskip} % Activate to begin paragraphs with an empty line rather than an indent
\ifx\pdfoutput\undefined
\usepackage{graphicx}
\else
\usepackage[pdftex]{graphicx}
\lstset{language=C++} 

%Доп. пакеты:
\usepackage{lipsum}
\usepackage{multicol}

 

\begin{document}
\twocolumn
\section{Политропы}
\begin{equation*}
\frac{dV}{V} + \frac{dP}{P} = \frac{dT}{T}
\end{equation*}
\begin{equation*}
n = \gamma = \frac{C_p}{C_v} \text{ — для адиабаты}
\end{equation*}
\begin{equation*}
PV^n = const
\end{equation*}
\begin{equation*}
TV^{n-1} = const
\end{equation*}
\begin{equation*}
n = \frac{C - C_P}{C - C_V}
\end{equation*}
\begin{equation*}
A = \frac{P_1V_1}{n - 1} \bigg(1 - \bigg(\frac{V_1}{V_2}\bigg)^{n-1}\bigg)
\end{equation*}
Теплоемкость смеси газов:
\begin{equation*}
C_{Vсмеси} = \frac{\sum \nu_i C_{V_i}}{\sum \nu_i}
\end{equation*}
Скорость звука/истечение газов:
\begin{equation*}
c_{зв} = \sqrt{\bigg(\frac{(\partial P}{\partial \rho}\bigg)_{ад}} = \sqrt{(\gamma - 1)C_P T} = 
\end{equation*}
\begin{equation*}
= \sqrt{\frac{\gamma P}{\rho}}= \sqrt{\frac{\gamma RT}{M}}
\end{equation*}
Энтропия:
\begin{equation*}
dS = \frac{\delta Q}{T}
\end{equation*}
\begin{equation*}
dS = C_V \frac{dT}{T} + R\frac{dV}{V}
\end{equation*}
\begin{equation*}
S(T, V) = \nu(C_v \ln T + R\ln V + S_0)
\end{equation*}
\begin{equation*}
S(T, P) = \nu(C_P \ln T - R\ln P + S_0)
\end{equation*}
\begin{equation*}
\Delta S = \nu(C_V \ln \frac{T}{T_0} + R \ln \frac{V}{V_0})
\end{equation*}
Циклы/машины:
\begin{equation*}
Q = \int TdS
\end{equation*}
\begin{equation*}
\oint \frac{\delta Q}{T} = 0
\end{equation*}
Приведенные теплоты:
\begin{equation*}
\frac{Q_н}{T_н} = \frac{|Q_х|}{T_х}
\end{equation*}

\section{Термодинамические потенциалы и соотношения}
Свободная энергия:
\begin{equation*}
\psi(T,V) = F(T,V) = U - TS 
\end{equation*}
Энтальпия:
\begin{equation*}
I(P,S) = H(P,S) = U + PV
\end{equation*}
Потенциал Гиббса:
\begin{equation*}
Ф(T,V) = G = Z = U - TS + PV
\end{equation*}
\begin{equation*}
d\psi = -PdV - SdT
\end{equation*}
\begin{equation*}
dФ = -SdT + VdP
\end{equation*}
\begin{equation*}
dI = TdS + VdP
\end{equation*}
\begin{equation*}
\bigg(\frac{\partial \psi}{\partial V}\bigg)_T = -P
\end{equation*}
\begin{equation*}
\bigg(\frac{\partial \psi}{\partial T}\bigg)_V = -S
\end{equation*}

Важнейшие соотношения:
\begin{equation*}
\bigg(\frac{\partial S}{\partial V}\bigg)_T = \bigg(\frac{\partial P}{\partial T}\bigg)_V
\end{equation*}
\begin{equation*}
\bigg(\frac{\partial S}{\partial P}\bigg)_T = -\bigg(\frac{\partial V}{\partial T}\bigg)_P
\end{equation*}

\begin{equation*}
\bigg(\frac{\partial T}{\partial V}\bigg)_S = -\bigg(\frac{\partial P}{\partial S}\bigg)_V
\end{equation*}

\begin{equation*}
\bigg(\frac{\partial T}{\partial P}\bigg)_S = \bigg(\frac{\partial V}{\partial S}\bigg)_P
\end{equation*}

\begin{equation*}
C_P - C_V = T\bigg(\frac{\partial P}{\partial T}\bigg)_V\bigg(\frac{\partial V}{\partial T}\bigg)_P
\end{equation*}
\begin{equation*}
P + \bigg(\frac{\partial U}{\partial V}\bigg)_T = T\bigg(\frac{\partial P}{\partial T}\bigg)_V
\end{equation*}

\begin{equation*}
\bigg(\frac{\partial P}{\partial T}\bigg)_V\bigg(\frac{\partial T}{\partial V}\bigg)_P\bigg(\frac{\partial V}{\partial P}\bigg)_T = -1
\end{equation*}
Тепловой коэффициент расширения:
\begin{equation*}
\alpha =
\frac{1}{V_0}\bigg(\frac{\partial V}{\partial T}\bigg)_P
\end{equation*}
Изотермическая сжимаемость:
\begin{equation*}
\beta_T = -\frac{1}{V}\bigg(\frac{\partial V}{\partial P}\bigg)_T
\end{equation*}
Изотермический модуль объемного сжатия:
\begin{equation*}
K_T = -V\bigg(\frac{\partial P}{\partial V}\bigg)_T
\end{equation*}
\section{Поверхностное натяжение}
\begin{equation*}
U_{пленки} = П(\sigma - T \frac{\partial \sigma}{\partial T}
\end{equation*}
\begin{equation*}
U_{пузыря} = 2П(\sigma - T \frac{\partial \sigma}{\partial T} + C_V V \Delta T
\end{equation*}
\begin{equation*}
F_{пов. натяжения} = \sigma l = ES \frac{2Пr - l}{l} \frac{\Delta l}{r}
\end{equation*}
\section{Фазовые переходы}
\begin{equation*}
Ф = const
\end{equation*}
\begin{equation*}
\frac{dP}{dt} = \frac{S_ж - S_т}{v_ж - v_т} = \frac{q_{пп}}{T(v_ж - v_т)}
\end{equation*}
При парообразовании:
\begin{equation*}
\frac{dP}{dt} = \frac{q}{Tv_{пара}} 
\end{equation*}
В окрестности тройной точки:
\begin{equation*}
q_{пл} + \lambda_{исп} - q_{возг} = 0
\end{equation*}
Увеличение массы насыщенного пара при увеличении $T$ на $\Delta T$:
\begin{equation*}
\Delta m = \frac{PVM}{RT^2}\bigg(\frac{\lambda M}{RT}-1\bigg)\Delta T
\end{equation*}

\section{Гидродинамика}
Формула Бернулли:
\begin{equation*}
\xi + \frac{P}{\rho} = const
\end{equation*}
\begin{equation*}
\frac{v^2}{2} + gh + \frac{P}{\rho} = const
\end{equation*}

\section{Течение газов}
\begin{equation*}
C_P T + \frac{v^2}{2} = C_P T_0
\end{equation*}
\begin{equation*}
c_{зв} = 
\end{equation*}

\section{Газ Ван-дер-Ваальса}
\begin{equation*}
P=\frac{RT}{V-b} - \frac{a}{V^2}
\end{equation*}
\begin{equation*}
U = C_v T - \frac{a}{V}
\end{equation*}
Политропа:
\begin{equation*}
T(V-b)^{n-1} = const
\end{equation*}
Адиабата:
\begin{equation*}
n = k = 1 + \frac{R}{C_V}
\end{equation*}

\begin{equation*}
\Delta S = C_V\ln \frac{T_2}{T_2} + R \ln \frac{V_2 - b}{V_1 - b}
\end{equation*}
Интегральный эффект Джоуля-Томсона:
\begin{equation*}
(C_V + R)(T_2-T_1) = \frac{bRT_1}{V_1-b} - \frac{2a}{V_1}
\end{equation*}

\onecolumn

\section{Элименты статистической термодинамики}

\section{Распределение Максвелла}
\subsubsection{ВЫТЕКАНИЕ/ВТЕКАНИЕ ГАЗА?}
????
\subsubsection{Распределение Максвела по модолю скорости:}\
\indent То есть доля числа частиц имеющих модуль скорсоти между \textbf{U} и \textbf{U + dU} выраженную через функцию распределения \textbf{F(U)}:

\begin{enumerate}
\item Трехмерный случай:
\[\dfrac{dn}{n_0} = f(U_x) \cdot f(U_y) \cdot f(U_z) dU_x\cdot dU_y \cdot dU_z = \Big(\dfrac{m}{2\Pi*k*T}\Big)^{3/2}*e^{\Big(-\dfrac{mU^2}{2kT}\Big)}\cdot 4\Pi*U^2dU \] 

\item Двумерный случай:
\[\dfrac{dn}{n_0} = \Psi(U)dU = \dfrac{m}{kT}\cdot e^{\Big(-\dfrac{mU^2}{2kT}\Big)}*U dU\]

\end{enumerate}
\section{Общее}
\begin{multicols}{2}[
\textcolor{blue}{\fbox{Общие полезные формулы(ДОПИСАТЬ!!)}} 
]
\begin{itemize}
\item Для иделаьонго газа:
\[<U> = \Big(\dfrac{8kT}{\pi m} = \dfrac{8RT}{\pi M}  \Big) \]
\end{itemize}

\begin{itemize}
\item Для иделаьонго газа:
\[<U> = \Big(\dfrac{8kT}{\pi m} = \dfrac{8RT}{\pi M}  \Big) \]
\end{itemize}

\end{multicols}


\section{Полезные интегралы:}

\numberwithin{equation}{section}
\begin{equation}
\int\limits_{-\infty}^{+\infty} e^{-\gamma \cdot x^2} dx = \sqrt{\frac{\Pi}{\gamma}}
\end{equation}

\begin{equation}
\int\limits_{-\infty}^{+\infty} x^2 \cdot e^{-\gamma \cdot x^2} dx = \frac{1}{2\gamma} \cdot \sqrt{\frac{\Pi}{\gamma}}
\end{equation}

\begin{equation}
\int\limits_{0}^{+\infty} x \cdot e^{-\gamma \cdot x^2} dx = \frac{1}{2\gamma} 
\end{equation}

\begin{equation}
\int\limits_{0}^{+\infty} x^3 \cdot e^{-\gamma \cdot x^2} dx = \frac{1}{2\gamma^2} 
\end{equation}

\begin{equation}
\int\limits_{0}^{+\infty} x^5 \cdot e^{-\gamma \cdot x^2} dx = \frac{1}{\gamma^3} 
\end{equation}



\end{document}