\documentclass[12pt]{article}
\usepackage[a4paper]{geometry}
                		% See geometry.pdf to learn the layout options. There are lots.
\geometry{a4paper}
\usepackage{listings}
\usepackage[cm]{fullpage}
\usepackage{gensymb}
\usepackage{layout}
\usepackage{amssymb,amsmath,amsfonts,latexsym,dsfont}
\usepackage{ upgreek }
\usepackage{xcolor}
\usepackage{titlesec}
\usepackage[warn]{mathtext}
\usepackage[T1,T2A]{fontenc}
\usepackage[utf8]{inputenc}
\usepackage{fancyhdr}
\usepackage{pgfplots, pgfplotstable}
\usepackage[english,bulgarian,ukrainian,russian]{babel}
%\titleformat{\section}[block]{\color{black}\Large\bfseries\filcenter}{}{1em}{}
\setcounter{secnumdepth}{0}
\renewcommand{\le}{\leqslant} 
\renewcommand{\ge}{\geqslant }

           		% ... or a4paper or a5paper or ... 
%\geometry{landscape}                		% Activate for rotated page geometry
%\usepackage[parfill]{parskip}    		% Activate to begin paragraphs with an empty line rather than an indent



\title{Лабораторная работа 2.1.6. Эффект Джоуля-Томсона.}
\author{Никита Павличенко}
%\date{October 2017}

\usepackage{natbib}
\usepackage{graphicx}

\pgfplotstableread{
X Y
411879.3 3.832923833
371868.168 3.415233415
337152.627 3.046683047
292434.303 2.555282555
251246.373 2.113022113
}\roomtemp

\pgfplotstableread{
X Y
417763.29 3.025404157
370691.37 2.586605081
342448.218 2.309468822
297729.894 1.963048499
245362.383 1.524249423
}\fiftytemp

\pgfplotstableread{
X Y
414821.295 2.628062361
372456.567 2.249443207
344801.814 2.026726058
296553.096 1.670378619
239478.393 1.269487751
}\seventytemp

\begin{document}

\maketitle
\fancyfoot[C]{МФТИ}%
\thispagestyle{fancy}
\newpage
\section{Цель работы}
\begin{enumerate}
    \item Определение изменения температуры углекислого газа при протекании через малопроницаемую перегородку при разных начальных значениях давления и температуры.
    \item Вычисление по результатам опытов коэффициентов Ван-дер-Ваальса $a$ и $b$.
\end{enumerate}
\section{В работе используются:}
\begin{enumerate}
    \item Трубка с пористой перегородкой;
    \item труба Дьюара;
    \item термостат;
    \item термометры;
    \item дифференциальная термопара;
    \item микровольтметр;
    \item балластный баллон;
    \item манометр.
\end{enumerate}

\section{Теоретическая часть}

\begin{equation}
    \varkappa = \frac{Q}{T_1 - T_2}\frac{1}{2\pi L}\ln\frac{r_2}{r_1}, 
\end{equation}
где $r_1$ — радиус нити, $r_2$ — радиус внешнего цилиндра, $L$ — длина нити.

\section{Ход работы}
\begin{enumerate}
    \item Найдем значение $\Delta T$ при разных давлениях внутри сосуда и разных температурах.\newline
    \begin{center}
    		\begin{tabular}{|c|c|c|c|c|c|}
    		\hline 
    		\multicolumn{6}{|c|}{$T=27,14C^\circ = 300,29K$} \\ 
    		\hline 
   		 $P$, кгс & 4,2 & 3,792 & 3,438 & 2,982 & 2,562 \\ 
   		 \hline 
    		$\Delta T$, $C^\circ$ & 3,833 & 3,415 & 3,047 & 2,555 & 2,113 \\ 
    		\hline 
    		$U$, мкв & 0,156 & 0,139 & 0,124 & 0,104 & 0,086 \\ 
    \hline 
    \end{tabular}
    \end{center}
    \begin{center}
    		\begin{tabular}{|c|c|c|c|c|c|}
    		\hline 
    		\multicolumn{6}{|c|}{$T=50,05C^\circ = 323,2K$} \\ 
    		\hline 
   		 $P$, кгс & 4,26 & 3,78 & 3,492 & 3,036 & 2,502 \\ 
   		 \hline 
    		$\Delta T$, $C^\circ$ & 3,025  & 2,587 & 2,309 & 1,963 & 1,524 \\ 
    		\hline 
    		$U-U_0$, мкв & 0,131 & 0,112 & 0,1 & 0,085 & 0,066 \\ 
    		\hline 
    		$U$, мкв & 0,138 & 0,119 & 0,107 & 0,092 & 0,073 \\ 
    \hline 
    \end{tabular}
    \end{center}
    \begin{center}
    		\begin{tabular}{|c|c|c|c|c|c|}
    		\hline 
    		\multicolumn{6}{|c|}{$T=70C^\circ = 343,15K$} \\ 
    		\hline 
   		 $P$, кгс & 4,23 & 3,798 & 3,516 & 3,024 & 2,442 \\ 
   		 \hline 
    		$\Delta T$, $C^\circ$ & 2,628  & 2,249 & 2,027 & 1,67 & 1,269 \\ 
    		\hline 
    		$U-U_0$, мкв & 0,118 & 0,101 & 0,091 & 0,075 & 0,066 \\ 
    		\hline 
    		$U$, мкв & 0,138 & 0,119 & 0,107 & 0,092 & 0,073 \\ 
    \hline 
    \end{tabular}
    \end{center}
    \begin{center}
    	\begin{tikzpicture}
    	\begin{axis}[
	title = Зависимость $\Delta T$ от $P$,
	height = 300,
	width = 450,
	xlabel = {$P$, Па},
	ylabel = {$\Delta T$, K},
	legend style={
		at={(0.165,-0.5)},
		anchor=south
	}
	]
	%\legend{$27.14C^\circ$, $\pgfmathprintnumber{\pgfplotstableregressiona} \cdot x
%\pgfmathprintnumber[print sign]{\pgfplotstableregressionb}$, $50.05C^\circ$, $\pgfmathprintnumber{\pgfplotstableregressiona} \cdot x
%\pgfmathprintnumber[print sign]{\pgfplotstableregressionb}$}
	\addplot[only marks, blue] table {\roomtemp};
	\addlegendentry{$27.14C^\circ$}
	\addplot [dashed, blue] table[
    y={create col/linear regression={y=Y}}
	]  {\roomtemp};
	\addlegendentry{
$1.07\cdot 10^{-5} \cdot x - 0.58$}
	\addplot[only marks, red] table {\fiftytemp};
	\addlegendentry{$50.05C^\circ$}
	\addplot [dashed, red] table[
    y={create col/linear regression={y=Y}}
	]  {\fiftytemp};
	\addlegendentry{$8.66\cdot 10^{-6} \cdot x - 0.62$}
	\addplot[only marks] table {\seventytemp};
	\addlegendentry{$70C^\circ$}
	\addplot [dashed, black] table[
    y={create col/linear regression={y=Y}}
	]  {\seventytemp};
	\addlegendentry{%
$\pgfmathprintnumber{\pgfplotstableregressiona} \cdot x
\pgfmathprintnumber[print sign]{\pgfplotstableregressionb}$}
	\end{axis}
    	\end{tikzpicture}
    \end{center}
 	\item Найдем зависимость $\Delta T$ от $P$. Посчитаем коэффициенты зависимости и построим график для каждой температуры.
 	\item Полученный тангенс угла наклона является коэффициентом Джоуля-Томсона. Найдем коэффициенты $a$ и $b$ через формулу
 	\begin{equation}
 	\mu = \frac{\Delta T}{\Delta P} \approx \frac{\frac{2a}{RT}-b}{C_p}
 	\end{equation}
 	Получаем для температур $27.14C^\circ$ и $50.05C^\circ$
 	\begin{equation*}
 	a = 1.04438\frac{Н\cdot м^4}{моль^2}
 	\end{equation*}
 	\begin{equation*}
 	b = 525.829 \frac{см^3}{моль}
 	\end{equation*}
 	Для $50.05C^\circ$ и $70C^\circ$
 	\begin{equation*}
 	a = 1.04438\frac{Н\cdot м^4}{моль^2}
 	\end{equation*}
 	\begin{equation*}
 	b = 525.829 \frac{см^3}{моль}
 	\end{equation*}
\end{enumerate}
\end{document}.